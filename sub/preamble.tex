\documentclass{gshs_report}
% 아래의 함수를 사용하면 이미지 파일들을 같은 디렉토리 내에 images 라는 이름을 가진 폴더를 생성한 후, 그 폴더 안에 넣어 사용할 수 있습니다.
% 사용하고자 한다면 주석을 푸십시오.
\graphicspath{{images/}}
% 이곳에 필요한 별도의 패키지들을 적어넣으시오.
%\usepackage{...}
\usepackage{verbatim} % for commment, verbatim environment
\usepackage{spverbatim} % automatic linebreak verbatim environment
%\usepacakge{indentfirst}
\usepackage{tikz}
%\tikzset{
%	image label/.style={
%		every node/.style={
			%fill=black,
			%text=white,
%			font=\sffamily\scriptsize,
%			anchor=south west,
%			xshift=0,
%			yshift=0,
%			at={(0,0)}
%		}
%	}
%}
\usepackage{amsmath}
\usepackage{amsfonts}
\usepackage{amssymb}
\usepackage{float}
\usepackage{graphicx}
\usepackage{tabularx}
\usepackage{multirow}
\usepackage{booktabs}
\usepackage{longtable}
\usepackage{gensymb}
\usepackage{wrapfig}
% \usepackage{hyperref}
% \hypersetup{
%     colorlinks=true,
%     allcolors=blue,
%     linkcolor=blue,
%     filecolor=magenta,      
%     urlcolor=cyan,
% }

%\usepackage{subcaption}
%\usepackage{floatrow}
%\usepackage{pict2e}
%\usepackage[backend=biber,style=authoryear]{biblatex}
%\usepackage{biblatex}
\usepackage{pgfplots}
\pgfplotsset{
	compat=newest,
	label style={font=\sffamily\scriptsize},
	ticklabel style={font=\sffamily\scriptsize},
	legend style={font=\sffamily\tiny},
	major tick length=0.1cm,
	minor tick length=0.05cm,
	every x tick/.style={black},
}

\usetikzlibrary{shapes}
\usetikzlibrary{plotmarks}
\usepackage{listings}
\usepackage{hologo}
\usepackage{makecell}
\usepackage{color}
\lstset{
	basicstyle=\small\ttfamily,
	columns=flexible,
	breaklines=true
}

\citation
\bibdata

%: ----------------------------------------------------------------------
%:               보고서 정보를 입력하시오
% ----------------------------------------------------------------------
% 아래와 같은 command를 만들면 길이가 긴 용어를 간편하게 사용할 수 있습니다. 단, 이미 지정된 함수명들은 새로운 함수명으로 사용할 수 없습니다.
% 연, 월, 일은 보고서 제출 날짜에 맞게 수정하십시오.

\newcommand{\gshs}{Gyeonggi Science High School for the Gifted }

% \researchtype{기초} % 기초 / 심화
% \reporttype{중간/결과} % 중간 / 결과

\title{정보 엔트로피를 이용한 여러 조건에서의 \linebreak 최적의 확률 분포 계산 및 해석}

% \title{제목} % 제목 개행 시 \linebreak 사용. \\나 \newline 은 안됨.
% \englishtitle{English Title}% 제목 개행 시 \linebreak 사용. \\나 \newline 은 안됨.

\author[1] {21032 김희윤}
% \email[1]{}
\author[2] {21033 남도현}
% \email[2]{}
\author[3] {21052 배요한}
% \email[3]{}
% \email[1]{(author@email.address)} % 제 1 저자 이메일
% \author[2] {} % 제 2 저자명
% \email[2]{} % 제 2 저자 이메일
% \advisor{Teacher} % 지도교사명
% \advisorEmail{(teacher@email.address)} % 지도교사 이메일

%%%%%%%%%%%%%%%%%%%%%%%%%%%%%%%%%%%%%%%%%%%%%%%
%%%% researchtype이 '심화'일 경우에만 나타남 %%%%
% \professor{Professor} % 지도교수명
% \professorEmail{(professor@email.address)} % 지도교수 이메일
% %%%%%%%%%%%%%%%%%%%%%%%%%%%%%%%%%%%%%%%%%%%%%%%%
% \summitdate{2020}{01}{01} % 제출일 (연, 월, 일)
% \newtheorem{definition}{정의}
