\section{서론}

엔트로피란 흔히 '무질서도'라고 알려진 개념이다. 정확한 정의로는 열역학적으로 사용이 불가능한 에너지를 나타내는 상태 함수이다. 열역학 제 2법칙과 관련된 물리량으로 크게 열역학적 정의, 통계역학적 정의, 정보이론의 정의로 나뉜다.

열역학적 정의는 클라우지우스가 카르노 기관에 대해 탐구하며 정의하였다.
\begin{equation}
    dS = \frac{dQ}{T}
\end{equation}
통계역학에서 사용되는 엔트로피는 볼츠만이 정의하였으며 기체 입자 계 (ensemble)에서 특정 상태가 일어날 경우의 수 $\Omega$에 로그를 취한 값을 갖는다.
\begin{equation}
    S = k_{B} \ln{\Omega}
\end{equation}

이를 토대로 섀넌이 정보 엔트로피 또는 섀넌 엔트로피(Shannon Entropy)를 특정 사건이 생길 수 있는 평균적인 로그 확률로 정의하였다. 거시 상태를 점유하는 미시 상태의 수를 상대적인 확률로 취급한 것이며 깁스의 entropy formula에 의해 매개된다. 즉, 미지의 정보가 많아 확실성이 떨어지는 상태가 정보 엔트로피가 높은 상태가 된다. 결론적으로 세 엔트로피의 정의는 동일하다.

최대 엔트로피 원리란 시스템의 현재 지식 상태를 가장 잘 나타내는 확률 분포는 가장 큰 엔트로피를 갖는 확률 분포라는 것이다. 초기 상태에 대한 정보가 부족한 경우 모든 상태를 균일하게 탐색할 필요성이 있으며 그러한 확률 분포는 엔트로피가 최대가 되는 확률 분포와 같다. 이는 열역학 제 2법칙과도 연관이 있다.

본 연구에서는 특정 조건에서 엔트로피가 최대가 되는 확률 분포에 대한 탐색을 두 가지 방법으로 진행하였다. 첫 번째 방법은 라그랑주 승수법을 이용하여 확률 분포를 이론적으로 계산해 내었다. 두 번째 방법으로는 최적화 문제에 흔히 쓰이는 유전 알고리즘(Genetic Algorithm)을 이용하여 실험적으로 확률 분포를 얻었다.